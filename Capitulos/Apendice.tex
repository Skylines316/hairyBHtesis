% !TeX root = ../Main.tex
\documentclass[../Main.tex]{subfiles}
\begin{document}

% Las siguientes líneas están para que la numeración en el apéndice sea correcta - NO CAMBIAR.
% Usar estas en caso de utilizar la clase book, de lo contrario comentar
\renewcommand{\thesection}{A\arabic{section}}
\renewcommand{\thetable}{A\arabic{section}.\arabic{table}}
\counterwithin{table}{section}
\counterwithin{figure}{section}

% Usar estas en caso de utilizar cualquier clase que NO SEA book.
%\renewcommand{\thesubsection}{A\arabic{subsection}}
%\renewcommand{\thetable}{A\arabic{subsection}.\arabic{table}}
%\counterwithin{table}{subsection}
%\counterwithin{figure}{subsection}

%---------- Escribir desde aquí en adelante


\section{Simetría axial y estacionaria}

Asumamos un espacio axiosimétrico y estacionario. En este espacio la métrica se puede expresar de la siguiente manera:
%
\begin{equation}
\dd{s}^{2} = -V(\rho, z)\qty(\dd{t}-\omega\dd{\phi})^{2}+V(\rho, z)^{-1}\rho^{2}\dd{\phi}^{2}+\Omega(\rho, z)^{2}\qty(\dd{\rho}^{2}+\Lambda(\rho, z)\dd{z}^{2})
\end{equation}
%
Donde $V,\Omega, \Lambda$ son funciones que solo dependen de $\rho$ y $z$la cual en el caso especial en que el espacio es vacío, es decir $R_{ab}=0$ tenemos la siguiente ecuación para $\rho$:
%
\begin{equation}
    D^{a}D_{a}\rho=0 \quad D_{a}: \text{derivada covariantes en la 2D atravesada por } \rho \, z
\end{equation}
%
% \section{Logo Universidad de Concepción}
% \begin{figure}[h]
%     \centering
% \caption{UdeC logo}
%     \includegraphics[width=0.2\columnwidth]{./Images/escudo_unsaac.jpg}
%     \label{fig_logo2}
% \end{figure}


\biblio %Se necesita para referenciar cuando se compilan subarchivos individuales - NO SACAR
\end{document}