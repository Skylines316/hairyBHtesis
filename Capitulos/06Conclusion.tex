\documentclass[../Main.tex]{subfiles}
\begin{document}

\section{Matriz de Consistencia}

\begin{table}[!h]
\begin{center}
\begin{tabular}{| m{5em} | m{5em} | m{5em} | m{5em} | m{5em} |}
\hline
Formulación del Problema & Formulación de Objetivos & Formulación de Hipótesis & Justificación & Metodología \\
\hline \hline
\textbf{¿Como hallar la complejidad de un Oscilador Armónico con el formalismo de la matriz de covarianza?} & Comparar nuestros resultados obtenidos con los métodos de Fubini Study y de Nielsen para obtener la complexity de un Oscilador Armónico Cuántico. Destajar las ventajas del formalismo de la matriz de covarianza para obtener la complejidad de un oscilador armónico cuántico. & Llegar al mismo resultado que por los otros métodos & Busco mostrar que el formalismo de la matriz de covarianza es mucho más simple para poder realizar los cálculos, los cálculos se hacen mucho más simples en comparación con los métodos de Fubini-Study, y de Nielsen. Además mostrar la posibilidad de utilizar este formalismo para cálculos mucho más complejos. & No Experimental, Explicativa \\
\hline
\end{tabular}
\end{center}
\end{table}

\biblio %Se necesita para referenciar cuando se compilan subarchivos individuales - NO SACAR
\end{document}